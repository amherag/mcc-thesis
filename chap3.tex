%% This is an example first chapter.  You should put chapter/appendix that you
%% write into a separate file, and add a line \include{yourfilename} to
%% main.tex, where `yourfilename.tex' is the name of the chapter/appendix file.
%% You can process specific files by typing their names in at the 
%% \files=
%% prompt when you run the file main.tex through LaTeX.

\chapter{Sistemas de Tutoría Inteligente}

Un Sistema de Tutoría Inteligente (ITS) es cualquier programa computacional
que facilite el aprendizaje de un alumno al momento de interactuar con
éste sistema de tutorías. Por lo general se utilizan una variedad de
técnicas de computación inteligente e inteligencia artificial para
lograr esta facilitación.

Estos sistemas tienen el objetivo de proveer al usuario o estudiante
con instrucción inmediata y personalizada. De esta forma un ser humano
no necesita intervenir en el proceso de aprendizaje del alumno, de tal
forma que se puede brindar ayuda de una forma constante y
consistente. Los ITS tienen siempre la meta de eficientar el
aprendizaje y las formas en las que los alumnos llegan a éste. Hay
muchos casos en donde los ITSs han demostrado ser eficaces para
cumplir con su principal objetivo, tal como ActiveMath
\cite{melis2004activemath}, una herramienta basada en la web con el
objetivo de ayudar al alumno a aprender matemáticas, y AutoTutor
\cite{graesser2005autotutor}, un ITS en donde se simula a un tutor
humano que mantiene una conversación con el alumno en lenguaje
natural.


Los Sistemas de Tutoría Inteligente consisten en cuatro componentes
básicos, los cuales han sido definidos en consenso entre los
investigadores a través de los años \cite{nwana1990intelligent}
\cite{freedman2000links} \cite{nkambou2010advances}. Estas cuatro
partes básicas son las siguientes:

\begin{itemize}
  \item El modelo del dominio
  \item El modelo del estudiante
  \item El modelo del tutor
  \item El modelo de la interfaz de usuario
\end{itemize}

El modelo del dominio se encarga de determinar todos los posibles
pasos requeridos para resolver un problema. En otras palabras, este
modelo contiene todos los conceptos, reglas, y estrategias del
dominio que va a ser aprendido \cite{nkambou2010advances}.

El modelo del estudiante puede ser considerado como el núcleo de todo
sistema de tutoría inteligente, ya que es el que se encarga de evaluar
el estado cognitivo del estudiante, así como sus estados afectivos y
su evolución a través del proceso de aprendizaje. Conforme el
estudiante trabaja paso a paso a través del proceso de solución de
problemas en el ITS, el sistema lleva a cabo un proceso llamado
modelo de rastreo. En cualquier momento que el modelo del estudiante
se desvía del modelo del dominio, el sistema identifica esta
desviación y marca que un error ha ocurrido.

El modelo del tutor acepta información de los modelos del dominio y
y del estudiante y toma decisiones acerca de las estrategias y
acciones que debe tomar el sistema de tutoría inteligente. En
cualquier momento en el proceso de solución de problemas, el alumno
puede pedir dirección sobre qué hacer después, en relación con su
ubicación actual en el modelo. El sistema reconoce cuándo el alumno se
ha desviado de las reglas de producción del modelo y da
retroalimentación al usuario, resultando en un periodo más corto para
alcanzar el aprendizaje \cite{koedinger1997intelligent}. El modelo del
tutor puede contener varios cientos de reglas de producción que pueden
tener uno de dos estados, aprendido o no aprendido.

El componente de la interfaz de usuario integra tres tipos de
informacieon que son necesitados para llevar a cabo un diálogo:
conocimiento sobre patrones de interpretación (para entender a un
hablante) y acción (para generar pronunciación) entre diálogos;
conocimiento del dominio necesario para comunicar contenido; y
conocimiento necesitado para comunicar la intención \cite{padayachee1999intelligent}.

\section{Protoboard}
\label{protoboard}