%% This is an example first chapter.  You should put chapter/appendix that you
%% write into a separate file, and add a line \include{yourfilename} to
%% main.tex, where `yourfilename.tex' is the name of the chapter/appendix file.
%% You can process specific files by typing their names in at the 
%% \files=
%% prompt when you run the file main.tex through LaTeX.

\chapter{Método Propuesto}

Para la clasificación de una persona de acuerdo a sus emociones a
través de dinámica de tecleo y dinámica de ratón, se necesitó crear un
sistema de tutoría inteligente. Este sistema fue basado en uno ya
existente llamado Protoboard (ver Sección \ref{protoboard}). Para que
cumpliera con nuestras necesidades, se tuvieron que hacer
implementaciones adicionales, entre ellas: que fuera capaz de mostrar un curso
completo de algún lenguaje de programación, así como ser capaz de
calificar al alumno si pudo realizar sus ejercicios correctamente,
mostrar diferentes tipos de objetos de aprendizaje (por ejemplo, videos,
ejercicios y encuestas), y poder capturar las pulsaciones que
realizaba el usuario en el teclado, así como los movimientos y
pulsaciones de los botones del ratón.

Para entrenar diferentes modelos de clasificación se necesitó obtener
un dataset con características basadas en dinámica de tecleo y
dinámica de ratón, además de otras características, como el número de
intentos que necesitó un estudiante para llegar a la solución. Para
obtener estos datos se construyó un curso de programación de Python
para la plataforma de Protoboard, y se implementó un script en
Javascript para que registrara cada una de las pulsaciones de teclas
que realizara un usuario durante un ejercicio. Este script también
registra los movimientos del ratón, así como los botones que son
presionados.

La lista de las 39 características se presentan a continuación,
mostrando su nombre utilizado en los experimentos junto con una
descripción. Las características basadas en dinámica de tecleo fueron
inspiradas por el trabajo de Cleyton Epp
et. al. \cite{epp2011identifying}. En la Tabla \ref{features}, un
keydown significa el evento que ocurre cuando una tecla o botón es
presionado, y keyup es el evento que ocurre cuando éste es liberado.

\begin{center}
  \label{features}
  \begin{table}
  \scriptsize
  \begin{tabular}{| p{4cm} | p{10cm} |}
    \hline
    Nombre & Descripción \\ \hline
    2G\_1D2D\_MEAN & Tiempo promedio entre la 1ra (keydown) y 2da (keydown) tecla en un dígrafo\\ \hline
    2G\_1D2D\_SD & Desviación estándar de la característica anterior \\ \hline
    2G\_1Dur\_MEAN & Tiempo promedio entre el keydown y keyup de la 1ra
    tecla en un dígrafo \\ \hline
    2G\_1Dur\_SD & Desviación estándar de la característica anterior \\ \hline
    2G\_1KeyLat\_MEAN &  Tiempo promedio entre la 1ra (keyup) y el
    siguiente keydown en un dígrafo \\ \hline
    2G\_1KeyLat\_SD & Desviación estándar de la característica anterior \\ \hline
    2G\_2Dur\_MEAN & Tiempo promedio entre el keydown y keyup de la 2da
    tecla en un dígrafo \\ \hline
    2G\_2Dur\_SD & Desviación estándar de la característica anterior \\ \hline
    2G\_Dur\_MEAN & Tiempo promedio entre el primer keydown y el último
    keyup de un dígrafo \\ \hline
    2G\_Dur\_SD & Desviación estándar de la característica anterior \\ \hline
    2G\_NumEvents\_MEAN & Promedio del número de eventos contenidos en
    un dígrafo \\ \hline
    2G\_NumEvents\_SD & Desviación estándar de la característica anterior \\ \hline
    3G\_1D2D\_MEAN & Tiempo promedio entre la 1ra (keydown) y 2da
    (keydown) teclas de un trígrafo \\ \hline
    3G\_1D2D\_SD & Desviación estándar de la característica anterior \\ \hline
    3G\_1Dur\_MEAN & Tiempo promedio entre el keydown y keyup de la 1ra
    tecla en un trígrafo \\ \hline
    3G\_1Dur\_SD & Desviación estándar de la característica anterior \\ \hline
    3G\_1KeyLat\_MEAN & Tiempo promedio entre el keyup de la 1ra tecla y
    el siguiente keydown en un trígrafo\\ \hline
    3G\_1KeyLat\_SD & Desviación estándar de la característica anterior \\ \hline
    3G\_2D2D\_MEAN & Tiempo promedio entre la 2da (keydown) y 3ra
    (keydown) teclas en un trígrafo \\ \hline
    3G\_2D2D\_SD & Desviación estándar de la característica anterior \\ \hline
    3G\_2Dur\_MEAN & Tiempo promedio entre el keydown y keyup de la 2da
    tecla en un trígrafo \\ \hline
    3G\_2Dur\_SD & Desviación estándar de la característica anterior \\ \hline
    3G\_2KeyLat\_MEAN & Tiempo promedio entre el keyup de la 2da tecla y
    el siguiente keydown en un trígrafo\\ \hline
    3G\_2KeyLat\_SD & Desviación estándar de la característica anterior \\ \hline
    3G\_3Dur\_MEAN & Tiempo promedio entre el keydown y keyup de la
    última tecla en un trígrafo \\ \hline
    3G\_3Dur\_SD & Desviación estándar de la característica anterior \\ \hline
    3G\_Dur\_MEAN & Tiempo promedio entre el keydown de la 1ra tecla y
    el último keyup en un trígrafo \\ \hline
    3G\_Dur\_SD & Desviación estándar de la característica anterior \\ \hline
    3G\_NumEvents\_MEAN & Promedio del número de eventos contenidos en
    un trígrafo \\ \hline
    3G\_NumEvents\_SD & Desviación estándar de la característica anterior \\ \hline
    Mouse\_Presses\_Dur\_MEAN & Tiempo promedio entre el keydown y keyup
    de algún botón del ratón \\ \hline
    Mouse\_Presses\_Dur\_SD & Desviación estándar de la característica anterior \\ \hline
    Mouse\_Movement\_Dur\_MEAN & Tiempo promedio de todos los movimientos
    del ratón\\ \hline
    Mouse\_Movement\_Dur\_SD & Desviación estándar de la característica anterior \\ \hline
    Mouse\_Movement\_X\_MEAN & Promedio de la distancia recorrida en
    pixeles en la coordenada X por el ratón \\ \hline
    Mouse\_Movement\_X\_SD & Desviación estándar de la característica anterior \\ \hline
    Mouse\_Movement\_Y\_MEAN & Promedio de la distancia recorrida en
    pixeles en la coordenada Y por el ratón \\ \hline
    Mouse\_Movement\_Y\_SD & Desviación estándar de la característica anterior \\ \hline
    Attempts & Número de intentos requeridos para llegar a la
    solución del ejercicio \\ \hline
  \end{tabular}
  \caption{Tabla de características usadas}
  \end{table}
\end{center}

Para asociar estas características extraídas con una clase o etiqueta,
a los usuarios se les presenta una encuesta que sigue las
especificaciones del Método de Muestreo de Experiencia (ver Sección
\ref{esm}, en donde a los usuarios se les pregunta cuales son los
niveles de seis sentimientos que experimentaron durante sus intentos
por resolver el ejercicio anterior. De esta forma, un usuario si se
sentía muy frustrado, y para nada relajado, puede registrarlo en esta
encuesta. Se desarrollaron diez ejercicios, así que los usuarios
tienen que responder a diez encuestas si es que completan todo el
curso.

La plataforma de Protoboard puede ser accedida a través de
http://app.protoboard.org/.
Esta aplicación web se dejó
abierta para que cualquier estudiante que quisiera entrar a intentar
resolver el curso de Python pudiera hacerlo. Es importante notar que
los estudiantes no estaban obligados a terminar todo el curso.

Una vez que se extrajeron todos los datos, se comenzó con el
preprocesamiento de estos. Este preprocesamiento está basado en las
guías mencionadas en el trabajo por Clayton Epp
et. al. \cite{epp2011identifying}. Para ver una descripción más
detallada y técnica, uno se puede dirigir a la Sección
\ref{implementacion}. En general, se extrajeron un total de 39
características, 8 corresponden a dinámica de ratón, 30 a dinámica de
tecleo, y 1 que corresponde al número de intentos que necesitó el
usuario para poder llegar a la solución del ejercicio. El
preprocesamiento involucra obtener el promedio y la desviación
estándar de la diferencia entre los tiempos de los 100 dígrafos y
trígrafos más frecuentes dentro de los datos en el dataset. En el caso
del ratón, se obtiene el promedio y la desviación estándar de los
movimientos en las coordenadas X y Y dentro de la ventana del
explorador, así como la duración entre diferentes eventos de los
botones en el ratón, y la duración de los movimientos.

Un dígrafo, en este contexto, significa cualquier combinación de dos
teclas realizadas por parte del usuario con el teclado de
computadora. Un trígrafo es similar, pero siendo una combinación de
tres teclas.

Un paso adicional que se tomó es que los datos generados por la
encuesta del método de muestreo de experiencia fueron reducidos. En
lugar de manejar una clase nominal con cinco posibilidades (de muy en
desacuerdo a muy de acuerdo, ver Sección \ref{implementacion}), se
manejaron solamente tres (bajo, medio y alto, ó 0, 1 y 2).

Una vez preprocesados los datos, se crearon clasificadores basados en
redes neuronales (ver Sección \ref{ann}), árboles de decisión (ver
Sección \ref{j48}), clasificador bayesiano ingenuo (ver Sección
\ref{naive}), y k vecinos más cercanos (ver Sección \ref{knn}),
tomando estos datos como entrenamiento y prueba.

Se realizaron pruebas preliminares y se obtuvieron malos resultados
usando la totalidad de las características dentro del dataset. Por
esta razón se optó por atacar el problema desde la perspectiva de
encontrar un subconjunto de características que pudieran facilitar la
creación de un clasificador efectivo. En otras palabras, se buscaron,
dentro de las 39 características iniciales, cuáles de estas podían
funcionar mejor como datos de entrada para cada uno de los
clasificadores.

En la Sección \ref{eyr}, Experimentos y Resultados, se presentan los
subconjuntos de características que funcionaron mejor para cada uno de
los clasificadores, y para cada uno de los estados emocionales, así
como las precisiones y el coeficiente de Kappa que se obtuvieron. El
objetivo en cada uno de estos entrenamientos era lograr una precisión
cercana al 60\%, y un Kappa de por lo menos 0.21 o muy cercano a
éste. La justificación de éstos valores es que 60\% es considerado
bastante bueno en aplicaciones reales \cite{epp2011identifying}, y un
Kappa de 0.21 es considerado como suficiente \cite{landis1977measurement}.
