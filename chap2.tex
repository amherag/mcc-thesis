%% This is an example first chapter.  You should put chapter/appendix that you
%% write into a separate file, and add a line \include{yourfilename} to
%% main.tex, where `yourfilename.tex' is the name of the chapter/appendix file.
%% You can process specific files by typing their names in at the 
%% \files=
%% prompt when you run the file main.tex through LaTeX.

\chapter{Computación Afectiva}

Una de las áreas de estudio que se relacionan con el trabajo presentado en esta tesis es la computación afectiva, una rama de las ciencias computacionales que estudia cómo interpretar y emular las emociones humanas. Además de estar relacionada con las ciencias computacionales, la computación afectiva es un campo interdisciplinario que involucra a la psicología y a las ciencias cognitivas. Aunque una de las motivaciones más grandes de la computación afectiva es la emulación de la empatía \cite{diamond2003love} \cite{picard2000affective}, también se encarga del estudio de la emulación e interpretación de otras emociones. En este trabajo se ve la computación afectiva desde el enfoque de interpretación o detección de emociones por parte de un sistema de software.

Otros trabajos que han involucrado a la computación afectiva en ambientes de aprendizaje, desde la perspectiva de intepretación de emociones, son, por ejemplo, el trabajo de D'Mello, et al. \cite{sidney2005integrating} y el trabajo de Elliot, Rickel y Lester \cite{elliott1999lifelike}.

Existen dos enfoques principales que se han usado para describir a las emociones: categórico y dimensional. El enfoque categórico aplica etiquetas a diferentes estados emocionales a través del idioma \cite{ekman1992argument}. El enfoque dimensional usa dos ejes ortogonales llamados despertar y valencia \cite{russell2003core}. El despertar está relacionado con la energía del sentimiento y es típicamente descrito en un rango de bajo a alto despertar (por ejemplo, de somnoliento a emocionado). La valencia describe el placer (positivo) o aflicción (negativo) de un sentimiento. Las etiquetas de los diferentes estados emocionales pueden ser representados en este espacio bi-dimensional. Por ejemplo, el enojo sería un estado de alto despertar y baja valencia.

Ambos modelos de las emociones, el categórico y el dimensional, han sido usados en enfoques anteriores para identificar estados emocionales. Algunos enfoques usan características que son fácilmente reconocibles por otros humanos, tales como las expresiones faciales, los gestos, entonaciones vocales, y el idioma \cite{picard2000affective}. Por ejemplo, software para rastrear rostros ha sido usado para analizar expresiones faciales que han sido extraídas de imagenes de cámaras web, para inferir los estados emocionales de usuarios \cite{de2003real}. Este enfoque ha sido extendido a imágenes térmicas para identificar cambios en los patrones del flujo de sangre en la sangre que son sinónimos con diferentes expresiones faciales \cite{khan2006automated}.

Otros enfoques usan características que son menos reconocibles por otros humanos, pero pueden ser medidos con equipo especializado. Por ejemplo, se ha conducido investigación sobre la medición de los cambios psicológicos que ocurren en el cuerpo humano durante episodios emocionales usando sensores tales como respuesta galvánica de la piel, electromiografía del rostro, y la actividad del corazón \cite{fairclough2009fundamentals}.

\section{Fluidez}

La fluidez es un estado mental en el que un individuo, mientras efectúa una actividad, experimenta un sentimiento de enfoque o participación mental profunda, y disfruta el transcurso de la actividad \cite{csikszentmihalyi1991flow}. La fluidez está estrechamente relacionada con la motivación, sienda la diferencia principal que la motivación es el propósito o causa psicológica de cualquier acción \cite{schacter2010psychology}. De esta forma, una persona puede estar muy motivada a ejecutar y continuar con la ejecución de cierta acción, pero no necesariamente estar en un estado de fluidez.

De acuerdo a Csikszentmihalyi, la fluidez es una motivación completamente enfocada. Es una inmersión total y representa talvez la última experiencia en aprovechar las emociones al servicio de las acciones y el aprendizaje. En la fluidez, las emociones no están solamente contenidas y canalizadas, sino también son positivas, energizadas, y alineadas con la tarea siendo realizada en ese momento. La fluidez tiene muchas de las mismas características positivas de la hiperconcentración.

Nakamura y Csikszentmihalyi identificaron los siguientes seis factores como requerimientos para poder experimentar la fluidez \cite{nakamura200918}:

\begin{itemize}
  \item Concentración intensa y enfocada en el momento
  \item Fusión entre acción y conciencia
  \item Pérdida de auto-conciencia reflexiva
  \item Una sensación de control personal sobre la situación o actividad
  \item Alteración de la experiencia subjetiva del tiempo
  \item Sensación de que la actividad es intrínsicamente gratificante
\end{itemize}

Estos aspectos pueden aparecer independientemente entre ellos, pero sólo una combinación de ellos pueden constituir la experiencia de la fluidez.

En cualquier momento, existe una gran cantidad de información disponible a cada individuo. Los psicólogos han encontrado que la mente de un ser humano puede ocuparse en sólo una cierta cantidad de información a la vez. De acuerdo al estudio de Csikszentmihalyi de 1956, ese número es cercano a 126 bits de información por segundo. Esto puede parecer un número muy grande (y, por lo tanto, mucha información), pero las tareas simples diarias necesitan mucha información. El tan solo mantener una conversación ocupa cerca de 40 bits de información por segundo; eso es cerca de una tercera parte de la capacidad total de un ser humano \cite{csikszentmihalyi1992optimal}. Es por esto que una persona que mantiene una conversación no puede enfocarse demasiado en otras tareas.

En la mayor parte, las personas son capaces de decidir en qué es lo que quieren enfocar su atención. Sin embargo, cuando una persona se encuentra en el estado de fluidez, ésta persona se encuentra completamente enfocada en esta tarea y, sin tomar una decisión voluntariamente, pierde conciencia sobre cualquier otra cosa: el tiempo, la gente, distracciones, e incluso necesidades básicas fisiológicas. Esto ocurre porque toda la atención de la persona en el estado de fluidez se encuentra sobre la tarea presente; no hay más atención que pueda ser asignada \cite{csikszentmihalyi1992optimal}.

\section{Método de Muestreo de Experiencia}
\label{esm}

El método de muestreo de experiencia es una metodología en donde a una serie de participantes se les pide detenerse en ciertos intervalos de tiempo y que tomen notas sobre su experiencia en tiempo real. El punto es lograr que ellos registren eventos temporales, tales como sus sentimientos durante ese momento y lugar. Se les puede dar una bitácora con muchas páginas idénticas. Cada página tiene una escala psicométrica, preguntas abiertas, o cualquier cosa puede ser usada para determinar su condición en ese lugar y tiempo. Este método fue desarrollado por Larson y Csikszentmihalyi \cite{larson1983experience}.

Hay diferentes formas en las que a los participantes se les puede dar la señal de tomar notas en sus bitácoras, por ejemplo, usando alarmas preprogramadas. Un observador puede tener una alarma preprogramada idéntica, y de esta forma el observador puede registrar eventos específicos al mismo tiempo que los participantes se encuentran registrando sus sentimientos o comportamientos. La mejor opción es evitar que los participantes sepan de antemano cuándo se les pedirá que registren sus sentimientos, para que no puedan anticipar el evento, y que puedan de esta forma estar actuando naturalmente cuando se les pidan que se detengan y que toman notas sobre su condición actual.

La validación de estos estudios se logra a través de la repetición, para que de esta forma uno pueda buscar patrones, como observar que los participantes logran tener una mayor felicidad después de tener una comida. Estas correlaciones pueden ser después probadas por medio de otras técnicas para determinar causa y efecto, ya que el método de muestreo de experiencia solo muestra correlación.

