%% This is an example first chapter.  You should put chapter/appendix that you
%% write into a separate file, and add a line \include{yourfilename} to
%% main.tex, where `yourfilename.tex' is the name of the chapter/appendix file.
%% You can process specific files by typing their names in at the 
%% \files=
%% prompt when you run the file main.tex through LaTeX.

\chapter{Computación Afectiva}

Una de las áreas de estudio que se relacionan con el trabajo presentado en esta tesis es la computación afectiva, una rama de las ciencias computacionales que estudia cómo interpretar y emular las emociones humanas. Además de estar relacionada con las ciencias computacionales, la computación afectiva es un campo interdisciplinario que involucra a la psicología y a las ciencias cognitivas. Aunque una de las motivaciones más grandes de la computación afectiva es la emulación de la empatía \cite{diamond2003love} \cite{picard2000affective}, también se encarga del estudio de la emulación e interpretación de otras emociones. En este trabajo se ve la computación afectiva desde el enfoque de interpretación o detección de emociones por parte de un sistema de software.


\section{Fluidez}

\section{Método de Muestreo de Experiencia}

