%% This is an example first chapter.  You should put chapter/appendix that you
%% write into a separate file, and add a line \include{yourfilename} to
%% main.tex, where `yourfilename.tex' is the name of the chapter/appendix file.
%% You can process specific files by typing their names in at the 
%% \files=
%% prompt when you run the file main.tex through LaTeX.

\chapter{Conclusiones y Trabajo Futuro}

A continuación se presentan las conclusiones sobre el trabajo
presentado en este documento de tesis, así como el trabajo futuro, el
cual será realizado como parte de una tesis de doctorado.

\section{Conclusiones}

Los resultados obtenidos y mostrados en la Sección \ref{eyr}
demuestran que es posible crear clasificadores efectivos a partir de
las 39 características (ver Tabla \ref{features}) basadas en dinámica
de teclado y dinámica de ratón. Todos los experimentos arrojaron
resultados exitosos, a excepción de la red neuronal entrenada para
predecir Relajamiento, la cual tuvo un coeficiente de Kappa de
0.207. Sin embargo, éste clasificador puede ser mejorado aumentando el
número de generaciones o cambiando el momentum y la taza de
aprendizaje. En realidad, esta declaración es cierta para todos los
clasificadores generados, ya que en este trabajo sólamente se enfocó
en encontrar un subconjunto de características del conjunto propuesto
que funcionara mejor para cada uno de los clasificadores para cada
estado mental. De esta forma, cada clasificador puede ser mejorado de
otras formas, como optimizando sus parámetros.


\section{Trabajo Futuro}

Para una tesis de doctorado se planea trabajar con los seis estados
mentales presentados en esta tesis, para intentar conseguir una
experiencia de fluidez para el estudiante que interactúe con un curso
en un sistema de tutoría inteligente.

%* Lo que haré en la tesis de doctorado
%* Mejorar estos modelos
%* Adaptar la plataforma para que solo pregunte a veces
%* Recaudar más datos
%* Hacer más ejercicios
%* Presentar hints