% -*-latex-*-
% 
% For questions, comments, concerns or complaints:
% thesis@mit.edu
% 
%
% $Log: cover.tex,v $
% Revision 1.8  2008/05/13 15:02:15  jdreed
% Degree month is June, not May.  Added note about prevdegrees.
% Arthur Smith's title updated
%
% Revision 1.7  2001/02/08 18:53:16  boojum
% changed some \newpages to \cleardoublepages
%
% Revision 1.6  1999/10/21 14:49:31  boojum
% changed comment referring to documentstyle
%
% Revision 1.5  1999/10/21 14:39:04  boojum
% *** empty log message ***
%
% Revision 1.4  1997/04/18  17:54:10  othomas
% added page numbers on abstract and cover, and made 1 abstract
% page the default rather than 2.  (anne hunter tells me this
% is the new institute standard.)
%
% Revision 1.4  1997/04/18  17:54:10  othomas
% added page numbers on abstract and cover, and made 1 abstract
% page the default rather than 2.  (anne hunter tells me this
% is the new institute standard.)
%
% Revision 1.3  93/05/17  17:06:29  starflt
% Added acknowledgements section (suggested by tompalka)
% 
% Revision 1.2  92/04/22  13:13:13  epeisach
% Fixes for 1991 course 6 requirements
% Phrase "and to grant others the right to do so" has been added to 
% permission clause
% Second copy of abstract is not counted as separate pages so numbering works
% out
% 
% Revision 1.1  92/04/22  13:08:20  epeisach

% NOTE:
% These templates make an effort to conform to the MIT Thesis specifications,
% however the specifications can change.  We recommend that you verify the
% layout of your title page with your thesis advisor and/or the MIT 
% Libraries before printing your final copy.

\title{Análisis de Sentimientos usando Dinámica de Tecleo y Dinámica
  de Ratón para el Secuenciado de Ejercicios de Programación de Computadoras}

\author{Amaury Hernández Águila}
% If you wish to list your previous degrees on the cover page, use the 
% previous degrees command:
%       \prevdegrees{A.A., Harvard University (1985)}
% You can use the \\ command to list multiple previous degrees
%       \prevdegrees{B.S., University of California (1978) \\
%                    S.M., Massachusetts Institute of Technology (1981)}
\department{División de Estudios de Posgrado e Investigación}

% If the thesis is for two degrees simultaneously, list them both
% separated by \and like this:
% \degree{Doctor of Philosophy \and Master of Science}
\degree{Maestro en Ciencias Computacionales}

% As of the 2007-08 academic year, valid degree months are September, 
% February, or June.  The default is June.
\degreemonth{Julio}
\degreeyear{2014}
\thesisdate{30 de Julio, 2014}
\copyrightnoticetext{Tijuana, Baja California, México}

%% By default, the thesis will be copyrighted to MIT.  If you need to copyright
%% the thesis to yourself, just specify the `vi' documentclass option.  If for
%% some reason you want to exactly specify the copyright notice text, you can
%% use the \copyrightnoticetext command.  
%\copyrightnoticetext{\copyright IBM, 1990.  Do not open till Xmas.}

% If there is more than one supervisor, use the \supervisor command
% once for each.
\supervisor{Dr. José Mario García Valdez}{.}

% This is the department committee chairman, not the thesis committee
% chairman.  You should replace this with your Department's Committee
% Chairman.
\chairman{M.Cs. Alejandra Mancilla Soto}{.}

% Make the titlepage based on the above information.  If you need
% something special and can't use the standard form, you can specify
% the exact text of the titlepage yourself.  Put it in a titlepage
% environment and leave blank lines where you want vertical space.
% The spaces will be adjusted to fill the entire page.  The dotted
% lines for the signatures are made with the \signature command.
%\maketitle

\begin{titlepage}
  \begin{center}
    \includegraphics[width=0.75\textwidth]{./logos.png}~\\[1cm]
    \textsc{\LARGE Análisis de Sentimientos usando Dinámica de Tecleo y Dinámica
  de Ratón para el Secuenciado de Ejercicios de Programación de
  Computadoras}~\\[0.5cm]
\textsc{por}~\\[0.5cm]
\textsc{\Large Amaury Hernández Águila}~\\[2.0cm]

\textsc{\Large División de Estudios de Posgrado e Investigación}~\\[0.5cm]
\textsc{Tesis para obtener el grado de}~\\[0.5cm]
\textsc{\Large Maestro en Ciencias Computacionales}~\\[0.5cm]
\textsc{en el}~\\[0.5cm]
\textsc{\Large INSTITUTO TECNÓLOGICO DE TIJUANA}~\\[1cm]
\textsc{Julio 2014}~\\[0.2cm]
\textsc{Tijuana, Baja California, México}~\\[1cm]

\begin{minipage}{1.5\textwidth}
  \begin{flushleft} \large
    \emph{Director:}\\
    Dr. José Mario \textsc{García Valdez}
  \end{flushleft}
\end{minipage}

\begin{minipage}{1.5\textwidth}
\begin{flushleft} \large
\emph{Co-Directora:} \\
M.Cs. Alejandra \textsc{Mancilla Soto}
\end{flushleft}
\end{minipage}

\vfill

  \end{center}

\end{titlepage}

% The abstractpage environment sets up everything on the page except
% the text itself.  The title and other header material are put at the
% top of the page, and the supervisors are listed at the bottom.  A
% new page is begun both before and after.  Of course, an abstract may
% be more than one page itself.  If you need more control over the
% format of the page, you can use the abstract environment, which puts
% the word "Abstract" at the beginning and single spaces its text.

%% You can either \input (*not* \include) your abstract file, or you can put
%% the text of the abstract directly between the \begin{abstractpage} and
%% \end{abstractpage} commands.

% First copy: start a new page, and save the page number.
\cleardoublepage
% Uncomment the next line if you do NOT want a page number on your
% abstract and acknowledgments pages.
% \pagestyle{empty}
\setcounter{savepage}{\thepage}
\begin{abstractpage}
% $Log: abstract.tex,v $
% Revision 1.1  93/05/14  14:56:25  starflt
% Initial revision
% 
% Revision 1.1  90/05/04  10:41:01  lwvanels
% Initial revision
% 
%
%% The text of your abstract and nothing else (other than comments) goes here.
%% It will be single-spaced and the rest of the text that is supposed to go on
%% the abstract page will be generated by the abstractpage environment.  This
%% file should be \input (not \include 'd) from cover.tex.

Este trabajo presenta un método basado en dinámica de tecleo y dinámica de ratón para el análisis de los sentimientos de un estudiante mientras interactúa con un sistema de tutorías inteligente llamado Protoboard, el cual se enfoca en la enseñanza de programación de computadoras. Los datos generados por la dinámica de tecleo y de ratón podría utilizarse para recomendar una secuencia de ejercicios de programación para un estudiante que esté interactuando con el sistema. Esta secuencia de ejercicios podría afectar los estados mentales durante el transcurso de las lecciones y ejercicios de programación, con el propósito de mejorar la experiencia de aprendizaje del estudiante. El método se enfoca en afectar seis estados mentales: frustración, aburrimiento, relajación, distracción, concentración, y excitación. Hasta el momento, para este trabajo de tesis de maestría en ciencias computacionales, la investigación se concentró en predecir éstos estados mentales, usando redes neuronales para clasificar a un estudiante de acuerdo a su dinámica de tecleo y dinámica de ratón. La clasificación nos da como resultado diferentes niveles de cada estado mental. Como trabajo futuro, estos niveles se usarían como entradas a un sistema de recomendación para determinar una mejor secuencia de ejercicios para que sean presentados al estudiante.

\end{abstractpage}

% Additional copy: start a new page, and reset the page number.  This way,
% the second copy of the abstract is not counted as separate pages.
% Uncomment the next 6 lines if you need two copies of the abstract
% page.
% \setcounter{page}{\thesavepage}
% \begin{abstractpage}
% % $Log: abstract.tex,v $
% Revision 1.1  93/05/14  14:56:25  starflt
% Initial revision
% 
% Revision 1.1  90/05/04  10:41:01  lwvanels
% Initial revision
% 
%
%% The text of your abstract and nothing else (other than comments) goes here.
%% It will be single-spaced and the rest of the text that is supposed to go on
%% the abstract page will be generated by the abstractpage environment.  This
%% file should be \input (not \include 'd) from cover.tex.

Este trabajo presenta un método basado en dinámica de tecleo y dinámica de ratón para el análisis de los sentimientos de un estudiante mientras interactúa con un sistema de tutorías inteligente llamado Protoboard, el cual se enfoca en la enseñanza de programación de computadoras. Los datos generados por la dinámica de tecleo y de ratón podría utilizarse para recomendar una secuencia de ejercicios de programación para un estudiante que esté interactuando con el sistema. Esta secuencia de ejercicios podría afectar los estados mentales durante el transcurso de las lecciones y ejercicios de programación, con el propósito de mejorar la experiencia de aprendizaje del estudiante. El método se enfoca en afectar seis estados mentales: frustración, aburrimiento, relajación, distracción, concentración, y excitación. Hasta el momento, para este trabajo de tesis de maestría en ciencias computacionales, la investigación se concentró en predecir éstos estados mentales, usando redes neuronales para clasificar a un estudiante de acuerdo a su dinámica de tecleo y dinámica de ratón. La clasificación nos da como resultado diferentes niveles de cada estado mental. Como trabajo futuro, estos niveles se usarían como entradas a un sistema de recomendación para determinar una mejor secuencia de ejercicios para que sean presentados al estudiante.

% \end{abstractpage}

\cleardoublepage

\section*{Agradecimientos}

Primeramente quiero agradecer a mis padres por apoyarme durante todo
el transcurso de mi maestría.

%%%%%%%%%%%%%%%%%%%%%%%%%%%%%%%%%%%%%%%%%%%%%%%%%%%%%%%%%%%%%%%%%%%%%%
% -*-latex-*-
