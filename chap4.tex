%% This is an example first chapter.  You should put chapter/appendix that you
%% write into a separate file, and add a line \include{yourfilename} to
%% main.tex, where `yourfilename.tex' is the name of the chapter/appendix file.
%% You can process specific files by typing their names in at the 
%% \files=
%% prompt when you run the file main.tex through LaTeX.

\chapter{Características Basadas en Dispositivos de Entrada}

Un dispositivo de entrada es cualquier dispositivo periférico usado
para proveer datos y señales de control a algún sistema de
procesamiento de información, tales como una computadora. Los ejemplos
más comunes de estos dispositivos de entrada son teclado, ratón,
y cámaras digitales.

Muchos dispositivos de entrada pueden ser clasificados de acuerdo a:

\begin{itemize}
  \item La modalidad de la entrada (moción mecánica, audio, visual)
  \item Si el dispositivo de entrada es discreto (por ejemplo,
    pulsación de teclas) o continuo (por ejemplo, la posición de un ratón)
  \item El número de grados de libertad involucrados (por ejemplo, un
    ratón tradicional que trabaja en un espacio bidimensional)
\end{itemize}

La entrada directa casi siempre es necesariamente absoluta, pero la
entrada indirecta puede ser absoluta o relativa. Por ejemplo, las
tabletas para la digitalización de gráficos que no tienen una pantalla
embebida involucra una entrada indirecta y un sensado de posiciones
absolutas y con frecuencia corren en un modo absoluto de entrada, pero
también pueden ser configuradas para que puedan simular un modo de
entrada relativo tal como el de un touchpad, en donde una pluma stylus
puede ser levantada y reposicionada.

Un teclado es un dispositivo de interfaz humana que es representado
como una configuración de botones. Cada botón, o tecla, puede ser
usada como una entrada de un caracter lingüístico a una computadora, o
para llamar una función en particular de la computadora. Los teclados
tradicionales usan botones basados en resortes, aunque nuevas
variaciones usan teclas virtuales, o incluso proyecciones de teclados.

Un dispositivo apuntador es cualquier dispositivo de interfaz humana
que permite a un usuario introducir datos espaciales a una
computadora. En el caso de un ratón, esto es usualmente logrado a
través de la detección del movimiento a través de una superficie
física. Los dispositivos análogos, tales como los ratones en 3D o los
joysticks, funcionan reportando su ángulo de deflección. Los
movimientos de los dispositivos apuntadores son representados en la
pantalla a través de movimientos del apuntador, creando una forma
simple e intuitiva de navegar por una interfaz de usuario en la computadora.

\section{Dinámica de Tecleo}
\label{dinamica-de-tecleo}

El comportamiento biométrico de la dinámica de tecleo usa la forma y
el ritmo en el cual un individuo introduce caracteres mediante un
teclado o keypad \cite{deng2013keystroke} \cite{araujo2005user}
\cite{shepherd1995continuous}. Los ritmos del tecleo de un usuario son
medidos para desarrollar una plantilla de biométricas únicas de los
patrones de tecleo del usuario para futuras autenticaciones
\cite{panasiuk2010modified}. Las mediciones pueden estar disponibles a
través de casi cualquier teclado para ser registradas para determinar
el tiempo en el que determinada tecla fue presionada y el tiempo entre
la pulsación de esta tecla (key down) y su liberación (key up). Estos
tiempos registrados son después procesados a través de algún
algoritmo, el cual determina patrones para ser comparados en un futuro
\cite{chang2011user}. De forma similar, la información acerca de las
vibraciones puede ser utilizada para crear un patrón para un uso
futuro tanto en tareas de identificación como de autenticación.

Los datos necesarios para realizar un análisis de dinámica de tecleo
son obtenidos a través de un registro de pulsaciones. Normalmente,
todo lo que se registra cuando se realiza este proceso durante una
sesión, es la secuencia de caracteres correspondiente al orden en el
que estas teclas fueron presionadas por el usuario, y la información
de los tiempos en los que sucedieron estas pulsaciones.

Los investigadores están interesados en usar esta información de
dinámica de tecleo, la cual es normalmente descartada, para verificar
o incluso determinar la identidad de la persona que está produciendo
estas pulsaciones de teclas. Esto es frecuentemente posible porque
algunas características de la producción de tecleo son individuales
tales como la caligrafía o una firma.

Reglas muy sencillas pueden ser usadas para eliminar posibles usuarios
en casos simples. Por ejemplo, si sabemos que Juan escribe 20 palabras
por minuto, y la persona que está haciendo uso del teclado está
escribiendo a 70 palabras por minuto, podemos descartar a Juan como la
persona que está escribiendo actualmente. Esta forma de pruebas está
basada simplemente en la velocidad de escritura. Es solamente una
prueba de un solo camino, ya que siempre es posible que la gente vaya
más lento de lo normal, pero es inusual o imposible para ellos que
obtengan una velocidad el doble de su velocidad normal.

Ha habido muy poco trabajo sobre la aplicación de dinámica de tecleo a
la computación afectiva. Zimmerman et
al. \cite{zimmermann2003affective} describen un método para
correlacionar las interacciones del usuario (teclado y ratón) con un
estado emocional. Los estados emocionales son inducidos usando
videos. Sensores psicológicos fueron usados en conjunción con un
Maniquí de Auto-Valoración \cite{lang1980behavioral}, un método para
expresar subjetivamente estados emocionales. Los autores encontraron
diferencias significativas entre los estados neuronales y otros
estados emocionales, pero no pudieron distinguir entre los estados
inducidos.

Trabajo reciente por Vizer et al. \cite{vizer2009automated} usó
características basadas en el tiempo acerca de las pulsaciones de
teclado de un texto libre en conjunción con características
lingüísticas para identificar el estrés cognitivo y físico. Ellos
lograron una clasificación correcta del 62.5\% para el estrés físico y
un 75\% para el estrés cognitivo, lo cual ellos establecen que es
comparable con otras soluciones de computación afectiva. Ellos también
establecen que sus soluciones deberían de ser probadas con habilidades
variadas en escritura y diferentes teclados, con habilidades
cognitivas y físicas variadas, y en situaciones estresantes del mundo real.

En este trabajo se utiliza la dinámica de tecleo, en conjunto con la
dinámica de ratón, para la predicción de estados emocionales, lo cual
es un campo relativamente nuevo, ya que la mayor parte de la
investigación se ha centrado en la autenticación de usuarios.

\section{Dinámica de Ratón}
\label{dinamica-de-raton}

La dinámica de ratón describe el comportamiento de un individuo con un
dispositivo de computadora de apuntador, tal como un ratón o un
touchpad. Recientemente, la dinámica de ratón ha sido propuesta como
una biométrica de comportamiento, bajo la premisa de que el
comportamiento del ratón es relativamente único entre diferentes
personas. La dinámica de ratón no es la única biométrica de
comportamiento que ha sida propuesta basada en un dispositivo de
interacción humana. La dinámica de tecleo (Sección
\ref{dinamica-de-tecleo}), la cual mide los ritmos únicos de tecleo de
un individuo, ha sido el tema de una considerable cantidad de
investigaciones a lo largo de las últimas décadas y us uso para
autenticación ha mostrado buenos resultados
\cite{monrose2000keystroke}. Dado que una gran parte de la interacción
entre un humano y una computadora en estos días involucra a algún
dispositivo de apuntador, usar la dinámica de ratón para la
autenticación era un paso obvio.

En el contexto de la autenticación, la biométrica tiene varias
ventajas sobre las técnicas de autenticación tradiconales que
verifican la identidad sobre algo que un individuo conoce (por
ejemplo, una contraseña) o algo que tiene (por ejemplo, una tarjeta de
identificación). En particular, las biométricas no pueden ser
olvidadas o robadas. Adicionalmente, las biométricas basadas en los
dispositivos de interacción de humano-computadora tienen la ventaja de
que son menos molestas que las biométricas y no requieren un hardware
especializado para capturar los datos biométricos necesarios.

Mientras que la autenticación por medio de dinámica de tecleo ha sido
estudiada exhaustivamente durante las últimas tres décadas, la
dinámica de ratón ha ganado interés sólamente en la última década. En
general, la dinámica de ratón parece demostrar que es una buena
técnica de autenticación en las últimas propuestas recientes
\cite{ahmed2007new} \cite{gamboa2004behavioral}, las cuales reportan
tazas de error mejores o comparables a otras biométricas bien
establecidas, tales como reconocimiento de voz y rostro.

En este trabajo se utiliza la dinámica de ratón, en conjunto con la
dinámica de tecleo, para la predicción de estados emocionales, lo cual
es un campo relativamente nuevo, ya que la mayor parte de la
investigación se ha centrado en la autenticación de usuarios.