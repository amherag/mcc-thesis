% $Log: resumen.tex,v $
% Revision 1.1  93/05/14  14:56:25  starflt
% Initial revision
% 
% Revision 1.1  90/05/04  10:41:01  lwvanels
% Initial revision
% 
%
%% The text of your abstract and nothing else (other than comments) goes here.
%% It will be single-spaced and the rest of the text that is supposed to go on
%% the abstract page will be generated by the abstractpage environment.  This
%% file should be \input (not \include 'd) from cover.tex.
La computaci�n Evolutiva Interactiva (CEI) se utiliza en este trabajo con el fin de llevar a cabo la optimizaci�n de varios bloques de texto publicitarios. Los anuncios de texto siguen un formato similar al utilizado en una t�cnica llamada �Article Spinning�. Este formato permite que el algoritmo CEI evolucione el texto para un determinado grupo de personas, usando palabras y frases como partes variables que cambian de acuerdo a la evaluaci�n subjetiva de la gente que interact�a con el algoritmo. Despu�s de varias generaciones, el algoritmo de CEI da como resultado una versi�n del texto publicitario que, en teor�a, deber�a exhibir un incremento en rendimiento, de acuerdo a la funci�n de evaluaci�n subjetiva con la cual fue evolucionado. Para poder demostrar la eficiencia de los textos, �stos son comparados contra una versi�n determinada por un experto en alg�n campo relacionado a la mercadotecnia. Para esta comparaci�n, tres pruebas fueron realizadas: pruebas de memoria, reconocimiento, y persuasi�n. Los resultados obtenidos muestran que la CEI puede ser usada efectivamente para incrementar el impacto de un texto publicitario, pero m�s experimentos necesitan ser realizados.