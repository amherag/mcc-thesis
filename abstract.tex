% $Log: abstract.tex,v $
% Revision 1.1  93/05/14  14:56:25  starflt
% Initial revision
% 
% Revision 1.1  90/05/04  10:41:01  lwvanels
% Initial revision
% 
%
%% The text of your abstract and nothing else (other than comments) goes here.
%% It will be single-spaced and the rest of the text that is supposed to go on
%% the abstract page will be generated by the abstractpage environment.  This
%% file should be \input (not \include 'd) from cover.tex.

Este trabajo presenta un método basado en dinámica de tecleo y dinámica de ratón para el análisis de los sentimientos de un estudiante mientras interactúa con un sistema de tutorías inteligente llamado Protoboard, el cual se enfoca en la enseñanza de programación de computadoras. Los datos generados por la dinámica de tecleo y de ratón podría utilizarse para recomendar una secuencia de ejercicios de programación para un estudiante que esté interactuando con el sistema. Esta secuencia de ejercicios podría afectar los estados mentales durante el transcurso de las lecciones y ejercicios de programación, con el propósito de mejorar la experiencia de aprendizaje del estudiante. El método se enfoca en afectar seis estados mentales: frustración, aburrimiento, relajación, distracción, concentración, y excitación. Hasta el momento, para este trabajo de tesis de maestría en ciencias computacionales, la investigación se concentró en predecir éstos estados mentales, usando redes neuronales para clasificar a un estudiante de acuerdo a su dinámica de tecleo y dinámica de ratón. La clasificación nos da como resultado diferentes niveles de cada estado mental. Como trabajo futuro, estos niveles se usarían como entradas a un sistema de recomendación para determinar una mejor secuencia de ejercicios para que sean presentados al estudiante.
